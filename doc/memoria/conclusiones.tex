A lo largo de esta memoria hemos descrito el desarrollo de una aplicación que permite a partir de un fichero XML que contiene una plantilla basada en DICOM-SR generar una aplicación Android para este informe médico estructurado.\par
En este apartado resumiremos el estado del proyecto y las conclusiones a las que hemos llegado.\medskip\par

Nuestro objetivo es que la aplicación final simplificara el proceso de introducir informes médicos estructurados.
Para conseguirlo aunamos el desarrollo software con el diseño de una interfaz de usuario comprensible para el personal clínico que hará uso de la misma. Creamos una interfaz  de usuario limpia y coherente siguiendo esquemas de diseño  estructurado basados en el estilo internacional. \par
Un diseño correcto de la interfaz de usuario aprovechando las posibilidades que nos ofrecen las tabletas Android permite que el usuario sea más propenso a utilizar la aplicación y si a esto le unimos una arquitectura eficiente hemos conseguido crear una experiencia de usuario satisfactoria.\medskip\par

Durante el tiempo de trabajo del PFC hemos conseguido desarrollar: el esqueleto de una aplicación Android y una aplicación generadora de código automática que permite instanciar aplicaciones Android con los datos de un informe estructurado.\par
Aunque nos quedan cosas que mejorar y que ampliar, que detallaremos en el aparatado \ref{mejoras}, el proyecto está en un punto estable y cumple con los requerimientos que nos habíamos marcado.\medskip\par

Una de las partes más interesantes de este proyecto ha sido el de construirlo desde los cimientos. Lo que nos ha permitido aplicar conceptos de la ingeniería del software para planificar el desarrollo, de teoría de lenguajes para crear el analizador sintáctico, de algorítmica y programación orientada a objetos para desarrollar el software generador de código,\ldots En definitiva nos ha dado la posibilidad de poner en práctica múltiples conceptos teóricos vistos a lo largo de la carrera.\medskip\par

Por lo tanto hemos cumplido los objetivos en tanto que hemos sido capaces de desarrollar una aplicación generadora de código automático para Android que simplificará la obtención de informes médicos estructurados poniendo en práctica el conocimiento adquirido en la Ingeniería Informática.\par

