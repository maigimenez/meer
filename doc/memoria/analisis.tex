\section{Análisis de requisitos}

Afrontamos el desarrollo de la aplicación siguiendo las pautas del esquema tradicional de desarrollo de software en cascada. Por la naturaleza del problema alunas etapas se han reducido para poder dedicar más tiempo al desarrollo de la solución.\medskip\par
En el siguiente apartado describiremos los requisitos de la solución que hemos propuesto. Durante esta fase del proyecto analizaremos las necesidades que tienen los usuarios de nuestro sistema.\par
Muchos de los requisitos vienen definidos por el propio problema y el resto por la solución que proponemos y que hemos descrito a alto nivel en el apartado \ref{solucion}. La complejidad de este proyecto estriba en la propia generación de código a partir de un fichero DICOM-SR. Los casos de uso y los actores se reducen a la mínima expresión.\par
Recopilar y analizar los requisitos sienta las bases de nuestro proyecto.\par
\subsection{Requisitos funcionales}
\begin{itemize}
	\item A partir de un fichero con la plantilla de un informe médico en formato DICOM-SR como el fichero incluido en el apéndice \ref{dicom-sr-template} que pertenezca a una de las siguientes ontologías:
		\begin{itemize}
			\item 4: Exploración de mama
			\item 5: Mamografía
			\item 6: Ecografía.
			\item 7: Resonancia Magnética
		\end{itemize}
		Se generarán los ficheros necesarios para instanciar una aplicación Andoid para esta plantilla que tendrá soporte para 1 o 2 idiomas en función de los idiomas que soporte el fichero DICOM.
	\item Con los ficheros generados el usuario podrá introducirlos en el esqueleto de una aplicación Android. Obteniendo una aplicación Android funcional.
	\item El entorno para la ejecución del script que genera el código y el de la aplicación esqueleto de Android deben ser fácilmente replicables, para que el usuario pueda genererar sus propias aplicaciones en distintas plataformas. 
\end{itemize}
\subsection{Requisitos no funcionales}
\begin{itemize}
\item La aplicación que genera el código debe ser lo más genérica posible para así abarcar la mayor parte de informes DICOM-SR posibles.
\item La interfaz de la aplicación se construirá mediante piezas de código intercambiables, lo que permitirá al usuario modificar la interfaz para que se acople a su metodología de trabajo.
\item La interfaz debe ser simple e intuitiva. La interfaz se construirá siguiendo un sistema de cuadrícula dónde se dará especial importancia a la tipografía y al esquema de color, para que con los mínimos elementos consigamos una interfaz de usuario amigable. 
\item El usuario podrá configurar ciertos aspectos del script que genera al código de la aplicación Android. Como por ejemplo el directorio donde estos ficheros se almacenarán, cual es el lenguaje principal, \ldots
\item La aplicación deberá facilitar la introducción de datos 
\end{itemize}
\subsection{Requisitos de implementación}\label{req:implementacion}
\begin{itemize}
\item La aplicación que genere el código se desarrollará en Python 2.7.x y seguirá las guías de estilo PEP-8 para la escritura de código y PEP-257 para la documentación.
\item La aplicación final Android se desarrollará para la versión Ice Cream Sandwich (3.2) o superior. Deberá soportar tamaños de pantalla de  10 pulgadas o más y alta resoluciones.
\item La aplicación Android se ajustará a las recomendaciones de diseño que podemos encontrar en las guías para diseñadores y desarrolladores de la web de Android Developers.
\item Se encapsularán las librerías de las que dependa el proyecto haciendo uso de \emph{virtualenv}, que permite crear entornos de desarrollo para python.
\end{itemize}
\section{Actores}\label{sec:actores}
En la tabla \ref{actores}, podemos ver la definición formal de los usuarios que emplearán el sistema que vamos a desarrollar. \par
Únicamente tenemos un actor y un caso de uso, que veremos en el apartado \ref{sec:casos-uso}, porque la generación de código debe necesitar la mínima interacción de los usuarios. Sin embargo por coherencia con la metodología de desarrollo realizamos el modelado de los actores y de los casos de uso.\par

\begin{table}
\begin{center}

  
  \begin{tabular}{ |b{4cm}||b{3cm}|b{1cm}|b{2cm}|b{1.5cm}| b{1cm}| }
    \hline
    \cellcolor{RubineRed} {\color{White} Actor} & \multicolumn{2}{|l|}{Usuario}  & \multicolumn{2}{|l|}{\color{RubineRed} Identificador}  &  USER1 \\ 
    \hline \hline
    {\color{RubineRed} Descripción } & \multicolumn{5}{|l|}{Usuario que generará automáticamente aplicaciones Android.}  \\ 
    \hline
    {\color{RubineRed} Características } & \multicolumn{5}{|l|}{\parbox{10.5cm}{El usuario dispondrá del entorno python  y de las aplicación esqueleto de Android, así como de un fichero con la plantilla de un informe DICOM-SR.}} \\ 
    \hline
    {\color{RubineRed} Relaciones } & \multicolumn{5}{|l|}{}\\ 
    \hline
    {\color{RubineRed} Referencias } &  \multicolumn{5}{|l|}{USO1}\\ 
    \hline
    {\color{RubineRed} Autor } &  Mayte Giménez & {\color{RubineRed} Fecha } & 16/08/2013 & {\color{RubineRed} Version } & v1.0 \\ 
    \hline
    \end{tabular}
    \begin{tabular}{ |b{4cm}|b{6cm}|b{4.75cm}|}
    \hline
    \multicolumn{3}{|>{\columncolor{RubineRed}}l|}{\color{White}  Atributos} \\
    \hline
    {\color{RubineRed} Nombre } & {\color{RubineRed} Descripción } & {\color{RubineRed} Tipo} \\
    \hline
    Vacío & Vacío & Vacío \\
    \hline
    \end{tabular}
    \begin{tabular}{ |b{15.6cm}|}
    \hline
    \cellcolor{RubineRed} {\color{White} Comentarios}  \\
    \hline
	\parbox{15.5cm}{Aunque para ejecutar la aplicación el usuario no requiera ningún tipo de conocimiento del sistema generador de código o de los informes médicos, el usuario debe disponer de un fichero DICOM-SR correctamente elaborado. Además debe de tener la aplicación esqueleto Android para instanciar la aplicación final. Además debe disponer de un dispositivo Android con la versión 3.2 o superior.}    \\
	\hline

    \end{tabular}
\end{center}
	\caption{Actores del sistema}
	\label{actores}
\end{table}

\section{Casos de uso}\label{sec:casos-uso}
\renewcommand*{\arraystretch}{1.5}
%\begin{table}
\begin{center}
  \begin{longtable}{ |b{2.5cm}|b{4cm}|b{1cm}|b{2cm}|b{1.5cm}| b{2.5cm}| }
  
    \hline
    \cellcolor{RubineRed} {\color{White} Caso de uso} & \multicolumn{2}{|l|}{\parbox{4.5cm}{Configuración del generador de código}}  & \multicolumn{2}{|l|}{\color{RubineRed} Identificador}  &  CONF \\ 
    \hline \hline
    {\color{RubineRed} Actores } & \multicolumn{5}{|l|}{Usuario}  \\ 
    \hline
    {\color{RubineRed} Tipo } & \multicolumn{5}{|l|}{Primario}  \\ 
    \hline
    {\color{RubineRed} Referencias } & \multicolumn{5}{|l|}{}  \\ 
    \hline
    {\color{RubineRed} Precondición } & \multicolumn{5}{|l|}{\parbox{13cm}{El usuario dispone de un fichero para generar la aplicación que sigue el estándar DICOM-SR. Y tiene el entorno de desarrollo para la aplicación generadora de código preparado.} }  \\ 
    \hline
    {\color{RubineRed} Postcondición } & \multicolumn{5}{|l|}{\parbox{13cm}{Los ficheros de configuración del generador quedarán personalizados según las necesidades del usuario.}}  \\ 
    \hline
    {\color{RubineRed} Autor } &  Mayte Giménez & {\color{RubineRed} Fecha } & 16/08/2013 & {\color{RubineRed} Version } & v1.0 \\ 
    \hline
    {\color{RubineRed} Propósito } & \multicolumn{5}{|l|}{\parbox{13cm}{Configurar las opciones parametizables del generador de código de la aplicación Android.}}  \\ 
    \hline
    \multicolumn{6}{|l|}{{\color{RubineRed} Resumen }}  \\ 
    \hline
   	\multicolumn{6}{|l|}{\parbox{16cm}{ Existen un número de opciones que el usuario puede modificar como son: las cadenas de texto que aparecen en los idiomas soportados, la ruta dónde se creará el código generado automáticamente y las ontologías soportadas. Los ficheros de configuración permiten que el usuario edite estas opciones para que el código generado se adapte a sus necesidades.}}  \\ 
    \hline
    \multicolumn{6}{|l|}{\parbox{8cm}{{\color{RubineRed} Curso normal }}}  \\ 
    \hline
    1 & \multicolumn{2}{|l|}{\parbox{5cm}{ El usuario modifica alguno o varios de los parámetros de configuración}} &  & \multicolumn{2}{|l|}{}\\
    \hline
     & \multicolumn{2}{|l|}{} & 2 &  \multicolumn{2}{|l|}{\parbox{5cm}{El sistema lee la configuración y cuando el usuario ejecute el script de generación de código aplicará las modificaciones.}}\\
    \hline
    \multicolumn{6}{|l|}{\parbox{8cm}{{\color{RubineRed} Curso alterno }}}  \\ 
    \hline
    2b & \multicolumn{5}{|l|}{\parbox{13cm}{Si alguno de los parámetros de configuración no es válido, este volverá al valor por defecto.}}  \\ 
    \hline
    \caption{Casos de uso: Configuración del generador de código}
  	\label{tab:caso-uso:conf}
    \end{longtable}
\end{center}
%\end{table}

\begin{center}
  \begin{longtable}{ |b{2.5cm}|b{4cm}|b{1cm}|b{2cm}|b{1.5cm}| b{2.5cm}| }
    \hline
    \cellcolor{RubineRed} {\color{White} Caso de uso} & \multicolumn{2}{|l|}{\parbox{4.5cm}{Generación del código de la aplicación Android}}  & \multicolumn{2}{|l|}{\color{RubineRed} Identificador}  &  GENCOD \\ 
    \hline \hline
    {\color{RubineRed} Actores } & \multicolumn{5}{|l|}{Usuario}  \\ 
    \hline
    {\color{RubineRed} Tipo } & \multicolumn{5}{|l|}{Primario}  \\ 
    \hline
    {\color{RubineRed} Referencias } & \multicolumn{5}{|l|}{}  \\ 
    \hline
    {\color{RubineRed} Precondición } & \multicolumn{5}{|l|}{\parbox{13cm}{El fichero para generar la aplicación debe seguir el estándar DICOM-SR y pertenecer a las ontologías soportadas en la configuración.} }  \\ 
    \hline
    {\color{RubineRed} Postcondición } & \multicolumn{5}{|l|}{\parbox{13cm}{Se generarán los ficheros necesarios para instanciar la aplicación Android esqueleto.}}  \\ 
    \hline
    {\color{RubineRed} Autor } &  Mayte Giménez & {\color{RubineRed} Fecha } & 16/08/2013 & {\color{RubineRed} Version } & v1.0 \\ 
    \hline
    {\color{RubineRed} Propósito } & \multicolumn{5}{|l|}{\parbox{13cm}{ Generar parte del código necesario de una aplicación Android.}}  \\ 
    \hline
    \multicolumn{6}{|l|}{{\color{RubineRed} Resumen }}  \\ 
    \hline
   	\multicolumn{6}{|l|}{\parbox{16cm}{El usuario ejecutará el script pasándole el fichero DICOM-SR y el método de internacionalización de este fichero (i18n o default) y el sistema generará los ficheros de código necesarios para instanciar la aplicación Android.}}  \\ 
    \hline
    \multicolumn{6}{|l|}{\parbox{8cm}{{\color{RubineRed} Curso normal }}}  \\ 
    \hline
   	1 & \multicolumn{2}{|l|}{\parbox{5cm}{ El usuario ejecuta el script generador de código. Los argumentos serán el fichero DICOM-SR y el método de internacionalización}} & 2 & \multicolumn{2}{|l|}{\parbox{5cm}{El sistema comprueba que el fichero existe y que el método de internacionalización es correcto}}\\
    \hline
     & \multicolumn{2}{|l|}{} & 3 &  \multicolumn{2}{|l|}{\parbox{5cm}{El sistema realiza el análisis sintáctico del fichero DICOM-SR}}\\
    \hline
    & \multicolumn{2}{|l|}{} & 4 &  \multicolumn{2}{|l|}{\parbox{5cm}{El sistema carga en una estructura de árbol la información del fichero DICOM-SR}}\\
    \hline
    & \multicolumn{2}{|l|}{} & 5 &  \multicolumn{2}{|l|}{\parbox{5cm}{El sistema genera las cadenas de internacionalización necesarias para la aplicación Android}}\\
    \hline
    & \multicolumn{2}{|l|}{} & 6 &  \multicolumn{2}{|l|}{\parbox{5cm}{El sistema genera los ficheros XML para la interfaz de usuario.}}\\
    \hline
    & \multicolumn{2}{|l|}{} & 7 &  \multicolumn{2}{|l|}{\parbox{5cm}{El sistema genera las clases Java que modelan el informe médico}}\\
    \hline
    & \multicolumn{2}{|l|}{} & 8 &  \multicolumn{2}{|l|}{\parbox{5cm}{El sistema genera las clases Java que controlan la aplicación Android.}}\\
    \hline
    & \multicolumn{2}{|l|}{} & 9 &  \multicolumn{2}{|l|}{\parbox{5cm}{El sistema genera el manifiesto de la aplicación Android.}}\\
    \hline
    \multicolumn{6}{|l|}{\parbox{8cm}{{\color{RubineRed} Curso alterno }}}  \\ 
    \hline
    2b & \multicolumn{5}{|l|}{\parbox{13cm}{Si el sistema detecta que el método de internacionalización no es correcto usará el método por defecto, descartando la información del segundo idioma si existiera}}  \\ 
        \hline
    5b & \multicolumn{5}{|l|}{\parbox{13cm}{Si los ficheros XML de las cadenas de internacionalización ya existen en la ruta especificada no se vuelven a generar.}}  \\ 
    \hline
    6b & \multicolumn{5}{|l|}{\parbox{13cm}{Si los ficheros XML de la interfaz de usuario ya existen en la ruta especificada no se vuelven a generar.}}  \\ 
	\hline
    7b & \multicolumn{5}{|l|}{\parbox{13cm}{Si los ficheros Java con el modelo ya existen en la ruta especificada no se vuelven a generar.}}  \\
	\hline
	8b & \multicolumn{5}{|l|}{\parbox{13cm}{Si los ficheros Java que controlan la aplicación Android ya existen en la ruta especificada no se vuelven a generar.}}  \\
	\hline
	9b & \multicolumn{5}{|l|}{\parbox{13cm}{Si el manifiesto Android ya existe en la ruta especificada no se vuelven a generar.}}  \\
	\hline
    \caption{Casos de uso: Generación de código}
  	\label{tab:casos-uso:gen}
    \end{longtable}
\end{center}

\begin{center}
  \begin{longtable}{ |b{2.5cm}|b{4cm}|b{1cm}|b{2cm}|b{1.5cm}| b{2.5cm}| }
    \hline
    \cellcolor{RubineRed} {\color{White} Caso de uso} & \multicolumn{2}{|l|}{\parbox{4.5cm}{Instanciación de la aplicación Android}}  & \multicolumn{2}{|l|}{\color{RubineRed} Identificador}  &  INSAND \\ 
    \hline \hline
    {\color{RubineRed} Actores } & \multicolumn{5}{|l|}{Usuario}  \\ 
    \hline
    {\color{RubineRed} Tipo } & \multicolumn{5}{|l|}{Primario}  \\ 
    \hline
    {\color{RubineRed} Referencias } & \multicolumn{5}{|l|}{}  \\ 
    \hline
    {\color{RubineRed} Precondición } & \multicolumn{5}{|l|}{\parbox{13cm}{El usuario dispone de los ficheros generados a partir de un fichero DICOM-SR y la estructura de la aplicación Android que los contendrá.} }  \\ 
    \hline
    {\color{RubineRed} Postcondición } & \multicolumn{5}{|l|}{\parbox{13cm}{Se crea una aplicación Android para el informe médico DICOM-SR a partir del cual se habían generado los ficheros.}}  \\ 
    \hline
    {\color{RubineRed} Autor } &  Mayte Giménez & {\color{RubineRed} Fecha } & 16/08/2013 & {\color{RubineRed} Version } & v1.0 \\ 
    \hline
    {\color{RubineRed} Propósito } & \multicolumn{5}{|l|}{\parbox{13cm}{Disponer de una aplicación Android que permita al usuario introducir los datos de informes DICOM-SR de la ontología instanciada.}}  \\ 
    \hline
    \multicolumn{6}{|l|}{\parbox{8cm}{{\color{RubineRed} Resumen }}}  \\ 
    \hline
   	\multicolumn{6}{|l|}{\parbox{8cm}{ Con el esqueleto de la aplicación Android para este proposito y todos los ficheros necesarios para personalizarla, el sistema instancia la aplicación Android para esta plantilla específica.}}  \\ 
    \hline
    \multicolumn{6}{|l|}{\parbox{8cm}{{\color{RubineRed} Curso normal }}}  \\ 
    \hline
    1 & \multicolumn{2}{|l|}{\parbox{5cm}{El usuario llama al script con la ruta raíz dónde se encuentran los ficheros generados y la ruta raíz de la aplicación Android.}} &  & \multicolumn{2}{|l|}{\parbox{5cm}{ El sistema comprueba que los ficheros generados y la aplicación Android esqueleto existan.}}\\
    \hline
     & \multicolumn{2}{|l|}{} & 3 &  \multicolumn{2}{|l|}{\parbox{5cm}{El sistema copia los ficheros en la ruta que les corresponde dentro de la aplicación Android.}}\\
    \hline
    \multicolumn{6}{|l|}{\parbox{8cm}{{\color{RubineRed} Curso alterno }}}  \\ 
    \hline
    2b & \multicolumn{5}{|l|}{\parbox{13cm}{Si alguna de las rutas no existe o falta algún fichero necesario la ejecución se aborta.}}  \\ 
    \hline
    \caption{Casos de uso: Instanciación de la aplicación Android}
  	\label{tab:casos-uso:android}
    \end{longtable}
\end{center}